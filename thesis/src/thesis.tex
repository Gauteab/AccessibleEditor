\documentclass[a4paper,english]{ifimaster}

\usepackage[utf8]{inputenc}
\usepackage{babel,duomasterforside}
\usepackage{hyperref}

\usepackage[backend=biber,style=authoryear]{biblatex}

\addbibresource{citations.bib}

\title{Master Thesis}
\subtitle{Structural Code Editing With Assistive Technologies}
\author{Gaute Berge}

\begin{document}

% \duoforside[dept={Department of Informatics},
% program={Informatics: Programming and System Architecture},
% long]

\frontmatter{}
% \chapter*{Abstract}

% \tableofcontents{}
% \listoffigures{}
% \listoftables{}

% \chapter*{Preface}

\mainmatter{}
% \part{Introduction}

% Donald Knuth often says smart stuff ~\parencite{Knuth:2007:CPA:1283920.1283929}.

% % \chapter{Motivation}
% \begin{itemize}
%     \item{Injuries}
%     \item{Try to find statistics}
% \end{itemize}

% \chapter{Background}

% \section{Assistive Technologies}

% \begin{itemize}
%     \item{What are assistive technologies?}
%     \item{Who uses them?}
% \end{itemize}

% \subsection{Eye Trackers}
% \begin{itemize}
%     \item{tobii}
%     \item{issues (instability, neck strain)}
% \end{itemize}

% \subsection{Screen Readers}
% Probably not a major part of this project, but should probably be considered.

% \subsection{Speech Recognition}
% \begin{itemize}
%     \item{Signal processing}
%     \item{Natural language processing}
% \end{itemize}



% \section{Programming By Voice}
% \begin{itemize}
%     \item{How is this achieved?}
%     \item{How does programming differ from normal dictation?}
%     \item{NLP vs command based}
%     \item{Different solutions}
%     \item{Programming with talon.}
% \end{itemize}

\section{TalonVoice}
Talon is a system that enables users to use a computer with alternative, hands-free input methods.
It supports controls using voice commands, noise input, and eye tracking.
The features of Talon are customizable and extendable through a python scripting api.
Through this api users can define custom actions that responds to certain noises, voice commands, or where the user is looking.
These actions can be very simple, such as emulating a sequence of keystrokes, but can also be arbitrary python functions
that can send commands to the operating system, or make network requests.
This section should serve as an overview of Talon and how it is used to make different features of a computer accessible to people with disabilities.

\subsection{Current state of the project}
There are two versions of Talon available; a public release and a private beta.
The public release (version 0.0.8.42) is only fully supported on macOS versions 10.11 through 10.14.
This distribution also works on newer versions of macOS, but you cannot use the built-in speech recognition engine.
In the private beta there is support for both newer version of mac os, as well as Windows and Linux.
This version also comes with improvements to the built-in speech recognition engine, and a rework of the current scripting api, including a domain specific scripting language that can be used in conjunction with python.

\subsection{Distribution and monetization}

\newpage
\section{Mouseless Code Editing}
Most programs where the user is to interact with text provides a similar editing scheme.
The text is laid out on screen with a thin cursor that indicates where the next character you type will be inserted.
The mouse can be used for navigating, selecting and reordering text.
Selecting single words, lines, and larger blocks of text can be done with double-click, triple-click, and holding and dragging the cursor respectively.
When a block of text is selected it can be moved by dragging it with the mouse.
Most of these actions can also be performed using just the keyboard.
By using a combination of keyboard modifiers and arrow/navigation keys, the user can jump, select and delete whole text objects such as words, lines and paragraphs.
This editing scheme has the advantage of being intuitive and easy to understand.
For many use cases however, this editing scheme leaves a lot to be desired in terms of efficiency.
This is especially true for code editing.

\textbf{Explain why this is the case?}

In the following section I will discuss a radically different editing scheme and how this affects users of assistive technologies.

\subsection{Vim}
Vim is an originaly terminal based text editor available on most platforms.
It is the successor of the "vi" editor and the name stands for "vi-improved".
Vim is a modal editor which means that the editing scheme is centered around different modes.

\begin{center}
\begin{tabular}{|c|c|}
    Normal mode&a\\
    Visual mode&\\
    Select mode&\\
    Insert mode&\\
    Cmdline mode&\\
    Ex mode&\\
\end{tabular}
\end{center}


% \section{Code Editors}
% \begin{itemize}
%     \item{Difference between text editor and code editor}
%     \item{Plain and rich text formats}
%     \item{IDEs}
%     \item{Modal Editors}
% \end{itemize}


% \part{The project}

% \chapter{Planning the project}

% \part{Conclusion}

% \chapter{Results}

% \backmatter{}
% \printbibliography

\end{document}
