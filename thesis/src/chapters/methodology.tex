\documentclass[../thesis.tex]{subfiles}

\begin{document}

\chapter{Methodology And Evaluation}\label{methodology_and_evaluation}
I might use a mix of quantitative and qualitative methods. (multi- strategy) % read creswell,2018
I could hold interviews with users of Talon to gather both types of data.

\paragraph{Speed Testing}
I can test the users programming speed with and without my system.
Here I must consider factors such as performance anxiety.
The programs they would be asked to dictate would have to be carefully crafted in order to have a consistent difficulty level
across the test. 
Should they dictate the same program twice, or two different? How can I be sure they are similarly difficult?
Should I change the order of the two tests?
This might well be a good measurement. I'll discuss why this paragraph.


\section{Qualitative}
I can gather qualitative data by having the user program using my system and ask them whether or not they find using it to be an improvement.
The disadvantage to this approach is that the system might have a learning curve which makes the initial impression
worse than it would have been over time.

\section{Quantitative}
One quantitative measure I could use is to analyse larger codebases to see how the predicted time to speak common identifiers change.
If I can count the number of syllables in a word I can compare the length of normal phrases need to produce a given function in a code base and compare
it to that of my system. This analysis can be weighted by the frequency of said identifiers.
The advantage of this approach is simplicity in that I don't depend on external users.
One interesting point here is to see how the result of this analysis relate to the results gathered from interviews.
\subsection{Data Collection}
Analyze the Elm implementation of real world app (https://github.com/rtfeldman/elm-spa-example)

\end{document}
