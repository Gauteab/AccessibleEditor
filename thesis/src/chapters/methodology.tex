\documentclass[../thesis.tex]{subfiles}

\chapter{Methodology and Evaluation}\label{methodology_and_evaluation}
This chapter outlines the research methods used in this project,
and how the study was designed and executed in order to evaluate effectiveness of the prototype.
The details of the study will then be presented, followed by
discussion of alternative methods that could have been used.

% I might use a mix of quantitative and qualitative methods. (multi- strategy) % read creswell,2018
% I could hold interviews with users of Talon to gather both types of data.

\section{Research Methods}
Quantitative research is an approach for ``testing objective theories by examining the relations between variables'', whereas qualitative research is concerned with ``exploring and understanding the meaning of individuals or groups in the context of a social or human problem''~\parencite{creswell2018}.

When choosing a research approach it is important to keep in mind the goal of the study.
The overall goal of the project is to improve the effectiveness of vocal programming by the use of modern programming language tooling.
The prototype should serve as proof of the effectiveness of the tools chosen to achieve this task.
In order to validate that this is in fact achieved, the system needed to be tested with actual users.
As will be discussed in \Vref{alternative_methods}, there are also quantitative measures that could be used which would make a multi strategy approach suitable.
The challenges with these however lead me to implement a more qualitative approach where I tried to understand the experiences of users attempting to use system for the first time.


% Explain what qualitative research is
% Explain what quantitative researchers
% Explain why I chose qualitative, refer to alt methods section

\subsection{Prototype}
Developing a prototype for the project served two main purposes.
Firstly, the prototype should show that the tools chosen for the implementation has the capabilities needed to achieve the desired results.
Secondly, it was used to determine the usefulness of the ideas presented in this paper.
To achieve the latter, the prototype needed to reach a certain level of completeness in order to faithfully represent the concept.
Therefore I aimed to develop a \textit{functional prototype}.
A functional prototype, also known as a \textit{technical prototype}, ``demonstrates user and computer interactions'' and it 
``may be syntactically complete or incomplete''~\parencite{jones2000rapid}.

\subsection{Usability Testing}
Usability testing is ``any of those
techniques in which users interact systematically with a product or system under
controlled conditions, to perform a goal-oriented task in an applied scenario, and
some behavioral data are collected''~\parencite{wichansky2000usability}.

The goal of these tests was to verify that users will be able to install the prototype
on their own systems and to see if it behaved as expected in their Talon environment, which will be different from
the one the prototype is developed in.
Additionally, I wanted to observe to what extent the user grasps, and resonates with the concepts introduced by the system.

To this end I chose to conduct moderated and exploratory user tests.
Exploratory tests validate the product and determine users mental models~\parencite{vasalou2004human}.
Moderating the tests gives need the ability to point out certain subtleties of the systems behavior
which I thought might be necessary in order to prompt them to reflect on the potential benefits of the system.

The test was conducted remotely.
This is partly necessary due to the ongoing COVID-19 pandemic, but it also has the
benefit of being able to test the system in the user natural environment~\parencite{vasalou2004human}.

\subsection{Interviews}
Semi-structured interviews was chosen as a method because they are  well  suited  for  the  exploration  of  the perceptions and opinions of  respondents regarding complex issues, and enable probing for more information and clarification~\parencite{louise1994collecting}.
The interviews conducted had a list of questions, some of which deal with background and practical information for the interview and subsequent user test.
Interviewees had the opportunity to bring up any other related issues they might want to talk about and I asked follow-up questions
if I deemed it relevant to the study.

\section{Participants and Recruitment}
Participants for this study was chosen based on the following inclusion criteria:
\textit{Participants must be somewhat experienced with using Talon}.
Testing the system with people who already use Talon should give a clearer impression
about the benefits of the system as unexperienced users will make more mistakes that are not 
related to the performance of the system.

In order to recruit participants, the aforementioned slack channel was used to reach users of Talon.
A message was posted in the \#talon channel accompanied by a short video demonstration of me using the system.
Interested users would then message me privately in order to schedule a time for the interview.

\section{The Study}
The study consisted of meetings with participants where we did both the interview and the user tests.
The meeting was divided into four parts.
I started by briefly introducing the project on the goals of my research, followed by an explanation of what we would be doing for
the rest of the session.
Participants then had an opportunity to share their thoughts on the project, after which I began the first round of questions.
The first set of questions mainly covers background information and demographics, as well as information that is useful to know
during the user test.
We then went on to install the system and started to use it.

% \subsection{Part 2: User Tests}
Participants are given a list of commands to try, while I will be observing
the output of the system on taking note of any unexpected behavior or errors.
It's not necessary that participants do the commands in the exact order suggested and they are free to experiment as the please.
I will however make sure that in the end they have tested all the functionality of the prototype.

Once the participant was finished experimenting with the system, I began the second round of questions
which went into their impressions and their thoughts about the potential of such a system.
The data collected here will be the main focus of the subsequent analysis, and can be used to
inform potential future design iterations.
The interview guide can be seen in \Vref{interview_guide}.

\subsection{Technical Details}
The meetings were conducted over Zoom, a popular digital conferencing platform.
Participants were given access to the repository containing all the code for the system
as well as user configurations for Talon and Vim.
The installation instructions also included information about how to add Elm to the list of known language in knausj\_talon.
They could then install the system using NPM and run it using Node.




% \begin{enumerate}[label*=Q1.\arabic*]
%     \item How long have you been programming?
%     \item How long have you been using Talon or other Voice Coding systems?
%     \item How much do you use Talon compared to keyboard and mouse in your day to day work?
%     \item Which speech engine are you using?
%     \item What command set are you using (is it baased on knausj)?
%     \item Are you familiar with Elm or any similar languages such as Haskell or SML?
%     \item Do you have experience using Vim?
%     \item What are some issues or challenges you experience when programming with Talon?
% \end{enumerate}
% \begin{enumerate}[label*=Q2.\arabic*]
%     \item In general, what is your impression of what you have seen during this session?
%     \item Were there any feature you thought was particularly useful?
%     \item Were there any behaviour you found to be strange or unintuitive?
%     \item Are there any features that are not implemented that you think could improve the system?
%     \item Do you feel like the system addresses some of the challenges you have been experiencing with vocal programming?
%     \item Based on what you have seen so far, would you be interrested in using a finished version of a system like this?
% \end{enumerate}


% "focus on discovering and understanding the experiences, perspectives, and
% thoughts of participants - that is, qualitative research explores the meaning, purpose
% or reality" (Harwell, 2011).

% I can gather qualitative data by having the user program using my system and ask them whether or not they find using it to be an improvement.
% The disadvantage to this approach is that the system might have a learning curve which makes the initial impression
% worse than it would have been over time.

\section{Ethical Considerations}%
\label{sec:ethical_considerations}
% did not ask about health, users are often disabled
All participants volunteered after seeing the post I made in the Talon Slack without being individually addressed.
As they were made aware, all responses were anonymized. % and no direct quotes will be used in the study.
While I did ask questions relating to their reliance on Talon (how much they use it compared to standard peripherals), 
I did not ask about \textit{why} they are using Talon as that might prompt them to share information about their health.
At no point did I store any personal information about participants, and they did have the opportunity to stay completely anonymous
as Slack does not require users to use their actual names as display names.



\section{Alternative Methods}\label{alternative_methods}
This section presents a few alternative methods that could have been used for this project.
Ultimately these methods were not chosen due to practical limitations within the scope of this thesis,
but could hopefully be used in future studies.
\paragraph{Coding Speed:}
As the goal of the project is to increase overall programming efficiency, measuring an increase in time to complete
certain coding tasks would provide the most reliable proof.
In such a study there are many factors to consider.
For example, the programs subjects would be asked to dictate would have to be carefully crafted in order to have a consistent difficulty level across the test.
Performance anxiety would also have to be considered when evaluating the results.
This could potentially effect a study of vocal programming more so than others due to some users being self-conscious while dictating.
The main reason this metric was not used in this study is the time required to make sure that users
are adequately competent using the new system with a high degree of proficiency.

% \paragraph{Speed Testing}
% I can test the users programming speed with and without my system.
% Here I must consider factors such as performance anxiety.
% Should they dictate the same program twice, or two different? How can I be sure they are similarly difficult?
% Should I change the order of the two tests?
% This might well be a good measurement. I'll discuss why this paragraph.

\paragraph{Evaluating Grammar Efficiency:}
Another quantitative method I considered using was to analyze larger code bases
to see if one can estimate with some degree of certainty how much better the voice grammar
produced by the system is compared to the base commands set.
This problem is however nontrivial, and could potentially
required the formalization of a somewhat sophisticated cost model for vocal programming grammars. %  (see \Vref{sec:future_work}).
Another practical issue with this approach is that it would require a much higher degree of completeness for the prototype
in order to give meaningful results.



% \begin{itemize}
%     \item User test, coding speed
%     \item analyze identifier cost
% \end{itemize}
% One quantitative measure I could use is to analyse larger codebases to see how the predicted time to speak common identifiers change.
% If I can count the number of syllables in a word I can compare the length of normal phrases need to produce a given function in a code base and compare
% it to that of my system. This analysis can be weighted by the frequency of said identifiers.
% The advantage of this approach is simplicity in that I don't depend on external users.
% One interesting point here is to see how the result of this analysis relate to the results gathered from interviews.
% \subsection{Data Collection}
% Analyze the Elm implementation of real world app (https://github.com/rtfeldman/elm-spa-example)
