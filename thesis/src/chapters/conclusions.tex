\documentclass[../thesis.tex]{subfiles}

\chapter{Conclusions and Future Work}%
\label{cha:conclusions_and_future_work}

In this thesis I have presented a working prototype
of an extension to Talon that can increase the efficiency of vocal programmers
, along with a user study to evaluate its effectiveness
in achieving this task.

\paragraph{}

The research questions were aimed at discovering
tools that can be used to make vocal programming
more efficient and high-level.
This was achieved using TreeSitter, which proved to be an excellent tool
for this task.
Particularly the Query API made it very easy to extract
relevant pieces of a program.
LSP was also demonstrated to show great promise, but was not integrated
into the prototype.

\paragraph{}

The response from users of Talon indicates that
the prototype was indeed useful in addressing
some of the challenges they were experiencing
with programming by voice.
Users were particularly impressed by the navigational features.
The structural editing capabilities
made editing less repetitive, and exposing
symbols to Talon made code dictation faster and more accurate.

\section{Future Work}%
\label{sec:future_work}
The field of vocal programming is severely underrepresented
in academic research.
Advancements in this field can significantly
improve the lives of the increasingly many
developers with RSI and/or physical disabilities.
This section outlines a few topics that would be interesting
to research further.

\paragraph{Comparison of vocal programming systems:}
Talon is not the only advanced voice control system available.
It would be very useful for researchers approaching the field
to have an overview of what is available
and what the differences are between the systems.
This could also help encourage collaboration
between the maintainers of these projects, which will
drive advancements in the field even further.

\paragraph{Structural editing with TreeSitter:}
The system presented in this thesis
used TreeSitter to implement structural editing commands
with voice control.
However as previously stated, the editing capabilities
were somewhat rudimentary and it would be interesting to see
how far this technology can be pushed to replicate the features
of structural editors in a normal text editor.
For example, the idea of ``holes''~\parencite{omar2019live} was not explored
in this thesis.

\paragraph{Effect on learning curve:}
While the system was designed primarily with expert users in mind,
as one participant noted, some of the features
might be even more beneficial for novice users.
This is because the high-level commands should be
easier to remember and more intuitive.
A study on the effects this extension has
on the learning curve for new users could help shape
future design.

% \section{Comparison of Voice Programming Systems}
% \section{Who benefits the most from structural navigation?}
% Noivice vs experts? One user noted they thing the system would benefit beginners a lot due to the reduced cognitive load.
% \section{Understanding users mental model of programs}
% How easy is it to grasp the idea of ``nodes'' in a program?
% \section{Voice Command Cost Model}\label{cost_model}
% generate grammar from grammar.json???
% Limitation: need more participants with less experience with programming

