\documentclass[../thesis.tex]{subfiles}

\begin{document}

\chapter{The Project}\label{the_project}
The goal of this project is to create a prototype of a system that can generate voice commands
for talon from a source file in a given program language to the increase accuracy and efficiency of programming.
The result should take the form of a server that can be easily integrated into any editor as a plug-in
which handles day generational voice commands and the communication with talon.
While the final version of the system should be general enough to handle any language, and be integratable with any the editor,
the prototype only aims to handle the \textit{Elm} programming language in the \textit{Neo-vim} editor.
This chapter will describe the design and architecture of the system.

\paragraph{Problem Description:}% concise description of the problem. maybe this should, earlier?
todo

\section{Voice Command Generation}%
\label{sec:voice_command_generation}
How to generate voice commands?
Cover how the spoken form of different identifiers will be.
Overview of different classes of identifiers (imports, aliases, functions, variables, types etc).
\subsection{Abbreviations}
peekCString->``P C S''. How does this relate to normal auto complete?


\section{Design Goals}%
\label{sec:design_goals}
configurable, accuracy, intuitive/consistent (user should not have to look up the generated commands), 
shortest commands possible

\section{Architecture}%
\label{sec:architecture}
editor<->my system<->talon

\section{Implementation}%
\label{sec:implementation}
implementation details such as programming language etc

\end{document}
