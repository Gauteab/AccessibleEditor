\documentclass[../thesis.tex]{subfiles}

\begin{document}

\chapter{Introduction}
Due to the prevalence of RSI in the tech industry, as well as the existence of other conditions that might limit a persons use of their arms/hands, the availability
of alternative input methods for programmers is extremely important.
Hands-free input technology enables people who would not otherwise be able to work on the computer to do so.
Enabling people to use standard applications such as web browsers, text editors, etc.\ is a complex problem in and of itself, and the domain of programming
introduces its own set of additional complexities.
Vocal programming, as coined by~\parencite{Arnold}, is the act of writing computer programs by dictating into the microphone as the primary input method as opposed to using a keyboard and mouse.
Programming differs from normal prose dictation in many subtitle ways such as higher frequency of special characters, and higher emphasis on editing text (and less emphasis on writing text).
Software developers also use a wider variety of applications and tools in their day-to-day work which all needs to be made accessible for vocal programmers.
Many of the solutions proposed by researchers involve a specialized environment for programming, such as a structure oriented editor.
As will be discussed in~\ref{scgp}, this approach comes with some trade-offs.
Recent advancements in programming language tooling, such as Microsoft's ``Language Server Protocol'', and GitHub's ``TreeSitter'' is bringing the power of IDEs to standard text editors.
In this paper I will use \textbf{Talon}, a general purpose system for hands-free input, to show how modern programming language
   tools can be leveraged to improve the workflow of vocal programmers in the programming environments they are already used to.

\section{Motivation}

\subsubsection{Repetitive Strain Injury}
% Here I will cover research related to injuries such as RSI in the tech industry.
% what is RSI?
Repetitive Strain Injury (RSI) is an umbrella term for a variety of pain syndromes caused by
performing repetitive tasks such as typing on a keyboard.
Symptoms can vary in severity, ranging from mild discomfort to complete disability.
Examples of commonly known RSIs include Carpal Tunnel Syndrome (also known as ``mouse arm'') and tendinitis.
% many people are afflicted, and need time off work
Every year, nearly 2 million US workers suffers from different forms of RSI, 600,000 of which
need time off work to recover from their injury (\textcolor{red}{ref}).
% http://ergonomictrends.com/rsi-statistics/ (need to find original references)
% this cost a lot of money
% treatments are not totally effective
% risk avoidance techniques such as posture, breaks and ergonomic keyboards
% why programmers are at high risk?
\subsubsection{Additional Risk Factors for Programmers}
In addition to simply being heavily dependent on the computer to do work,
programmers also use the keyboard somewhat differently from other office workers.
Most programming languages contain a lot more special symbols compared to what is used
normal English.
This fact necessitates more frequent use of a variety of combinations of modifier keys on the keyboard
which are not used as frequently in normal typing and are therefore less accessible.
Programmers must therefore move their hands more and differently.
``Emacs Pinky'' is the colloquial name for an RSI common amongst users of the Emacs editor.
This is a condition that causes pain from the elbow up to the pinky finger, and is coused
by a phenomenon called ``Ulnar deviation'' which happens when the hand is frequently rotated
outward.
This creates pressure on the ulnar nerve, and is exactly what happens when you reach for the control key on a standard keyboard,
which is a very frequent operation in Emacs.

\subsubsection{The Case for Hands-free Input Methods}
Tools for interacting with the computer other than keyboard and mouse
are known as alternative input methods.
These include things like digital pens, touchscreens, eye-trackers, foot pedals, et cetera.
Users with severe RSI or other motor impairments might need tools that do not require any physical motion.
Voice input, which will be the focus of this thesis, has very high potential for efficiency and can be made accessible to a wide variety of users.
This can be used both for dictating text as well as issuing commands.
\paragraph{}
It is my belief that using voice input for programming (``vocal programming'')
has extremely high potential not only for people with pre-existing conditions, 
but also for able-bodied programmers.
The goal of vocal programming systems should, in my view, go beyond
accessibility and aim to make programming more efficient and enjoyable for everyone
in order to create a future with significantly fewer cases of RSI.
% clearly there is a need for efficient ways of using a computer without the reliance on techniques made impossible by or introducing risk of RSI.
% motivate the goals voice systems should have. go beyond "accessible" name to become
% as efficient or more efficient

% \paragraph{Learning Curve}
% Here I explain breifly some of the difficulties people have when starting out with voice coding.
% Learning curve is the biggers, and is extasterbated by the fact that you need
% to be able to code in order to use the system profeciently, which leads to a 
% chicken-egg situation.
% \paragraph{Special Case vs. General Purpose}\label{scgp}

\newpage
\section{Research Questions}
\paragraph{Question 1:}
How can modern language tooling be leveraged to make vocal programming more efficient?
\paragraph{Question 2:}
How can the benefits of structural editors be achieved in text oriented environments for vocal programmers?

\section{Approach}
The way in which I will be attempting to answer these questions is to
investigate different tools used by code editors for increasing efficiency
when programming in a variety of different programming languages.
The main part of this project is developing a prototype for an extension
to Talon that will use these tools to make vocal programming more efficient.
Additionally, I will conduct a study to evaluate the usefulness and potential
of this prototype.

\section{Chapter Overview}
\paragraph{Chapter 2: \nameref{background}} gives an introduction to the relevant background for the project presented in this thesis.
This includes some information on vocal programming as well as technical background on the tools used in the project.
\paragraph{Chapter 3: \nameref{talon}} presents the vocal programming system chosen for this project
and serves to set the context for the work presented later on.
\paragraph{Chapter 4: \nameref{methodology_and_evaluation}} discusses possible research methods and explains
how data collection for this project will be carried out.
\paragraph{Chapter 5: \nameref{the_project}} covers the details of the project and explains
how the prototype will be built while discussing alternative methods, design goals
and additional features that are out of scope for this project.
\paragraph{Chapter 6: \nameref{results}} presents the results of the research.
\paragraph{Chapter 7: \nameref{future_work}} present some ideas for future research in this area.


\end{document}
