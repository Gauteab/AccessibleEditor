\documentclass[../thesis.tex]{subfiles}

\begin{document}

\chapter{Introduction}
Due to the prevalence of RSI in the tech industry, as well as the existence of other conditions that might limit a persons use of their arms/hands, the availability
of alternative input methods for programmers is extremely important.
Hands-free input technology enables people who would not otherwise be able to work on the computer to do so.
Enabling people to use standard applications such as web browsers, text editors, etc.\ is a complex problem in and of itself, and the domain of programming
introduces its own set of additional complexities.
Vocal programming, as coined by~\parencite{Arnold}, is the act of writing computer programs by dictating into the microphone as the primary input method as opposed to using a keyboard and mouse.
Programming differs from normal prose dictation in many subtitle ways such as higher frequency of special characters, and higher emphasis on editing text (and less emphasis on writing text).
Software developers also use a wider variety of applications and tools in their day-to-day work which all needs to be made accessible for vocal programmers.
Many of the solutions proposed by researchers involve a specialized environment for programming, such as a structure oriented editor.
As will be discussed in~\ref{scgp}, this approach comes with some trade-offs.
Recent advancements in programming language tooling, such as Microsoft's ``Language Server Protocol'', and GitHub's ``TreeSitter'' is bringing the power of IDEs to standard text editors.
In this paper I will use \textbf{Talon}, a general purpose system for hands-free input, to show how modern programming language
   tools can be leveraged to improve the workflow of vocal programmers in the programming environments they are already used to.

\section{Motivation}
\paragraph{Injuries}
Here I will cover research related to injuries such as RSI in the tech industry.
\paragraph{Learning Curve}
Here I explain breifly some of the difficulties people have when starting out with voice coding.
Learning curve is the biggers, and is extasterbated by the fact that you need
to be able to code in order to use the system profeciently, which leads to a 
chicken-egg situation.
\paragraph{Special Case vs. General Purpose}\label{scgp}

\section{Research Questions}
\paragraph{Question 1:}
Can the benefits of structural editors be achieved in text oriented environments for vocal programmers?
\paragraph{Question 2:}
Can modern language tooling be leveraged to make vocal programming more efficient?

\section{Approach}

\section{Chapter Overview}
\paragraph{Chapter 1:} \nameref{background} 


\end{document}
