\documentclass[../thesis.tex]{subfiles}

\chapter{Discussion}%
\label{cha:discussion}
This chapter discusses the results presented in Chapter~\ref{results}.
First, the results will be discussed and used to answer the research question.
Then there will be a discussion on the implications for theory and practice, followed by
a section on the limitations of this study.

\section{Increase in programming efficiency}%
\label{sec:increase_in_programming_efficiency}
The response to the prototype was overwhelmingly positive.
A positive impression does however not necessarily imply an increase in efficiency.

This section aims to answer the following research question:\newline
\textit{RQ1: How can modern language tooling be leveraged to make vocal programming more efficient?}

\paragraph{}

The types of improvements provided by the system presented in this thesis 
can be divided into two groups:
\begin{enumerate}
    \item Improvements to accuracy
    \item Improvements to interactions
\end{enumerate}
Improvements to accuracy (a reduction in the amount of misrecognition)
reduces the amount of time the user has to spend correcting errors.
Improvements to interactions allows users to accomplish more with fewer and shorter voice commands.
The rest of this section will discuss how the prototype
achieved this using modern language tooling.

\paragraph{Improvements to accuracy:}%
Voice command accuracy was improved by limiting the vocabulary
of certain commands based on the contents of the file being edited.
This was achieved by creating Talon lists for every
node type of interest which then can be used in place of
the normal dictation captures in voice commands.

The process of extracting identifiers was made very simple
due to the modern tools available.
For this task, TreeSitter Queries was used in the prototype, but Chapter~\ref{the_project}
also outlines how to do it using LSP.
% These tools are designed to work well with any programming language.


\paragraph{Improvements to interactions:}%
By making Talon aware of the symbols within the file, 
it was possible to issue much higher level commands for navigation and editing.
As mentioned by multiple participants, the standard way of navigating
is repetitive and low level.
The voice commands provided by the system allow users to distinguish
node based on their types as well as their associated identifiers, and perform actions on them in a single utterance.
This also tackles the problem of cognitive load by making the commands
easier to remember and more intuitive.

\paragraph{}

As seen, modern language tooling can be leveraged to make vocal programming
more efficient by exposing syntactic information about programs
to the voice control system in a way that enables higher-level commands
with better accuracy.



\section{Structural editing by voice}%
\label{sec:structural_editing_by_voice}
This section aims to answer the following research question:\newline
\textit{RQ2: How can the benefits of structural editors be achieved in text oriented environments for vocal programmers?}

\paragraph{}

The structural editing capabilities of the prototypes were quite rudimentary, but is already showing great promise.
The integration with TreeSitter makes it very easy to operate on the parse tree,
and enables complex transformations for many languages.
Integrating with TreeSitter can be done iva regular plug-ins
for editors support TreeSitter such as VSCode, or in an external program
as was done in the prototype.

Structural editing by voice even has some advantages over normal structural editing.
Being able to refer to a node by its name or type in a voice command
is often much more convenient then navigating to the node and then performing the action.


Structural editing was achieved in the prototype by simply providing an
API for acquiring the locations of nodes given a name and/or type that can be
called from Talon scripts.
Once the location was obtained, editing was performed by emulating keystrokes in the editor.


% One challenge with structural editing in text editors that have not been addressed
% is how to deal with missing parts of incomplete programs.
% TreeSitter's error tolerance makes it so that the system still works
% while there are syntax errors in the file, but I have not looked into




\section{Implications for Theory}%
\label{sec:implications_for_theory}
The capabilities of TreeSitter clearly extends far beyond the original
intended use case of syntax highlighting in code editors.
The very same core features that make it ideal for that use case
also make it an excellent choice for any other system
that deal with syntactic analysis of frequently changing programs.

Talon is an extremely flexible, general purpose voice control system.
The results of this study shows that it is easy to implement
fine-tuned domain specific voice control on top of Talon, as opposed to
making a specialized voice recognition system from scratch.

\section{Implications for Practice}%
\label{sec:implication_for_practice}
The prototype developed in this study is already very usable
for people using Vim and Elm.
I have shown that adding more languages should be very simple,
and the size of the Vim extension suggests that adding more editors
should also be relatively simple.

With a little extra effort, the work presented in this thesis
can address several issues that users of Talon are currently facing.



\section{Limitations}%
\label{sec:limitations}
There are several elements of the study that could be improved.

\paragraph{}

When conducting interviews to capture the experience of users it is recommended to have
between 10 and 30 participants~\parencite{griffin1993voice}.
My study only had 6 participants.
It should however be noted that many of the interviews yielded very similar results
which can indicate that the data set was already close to stagnating.
More importantly than the total number of participants, 
the study lacked participants with lower amount of experience
with programming in general.
Having more participants unfamiliar with
concepts like parse trees could have affected the results.

\paragraph{}

In this study, participants were only interviewed about
their initial impressions of the system.
Given that this is an expert system, it would be interesting to see
how users experience changes over time as they get more proficient.
If time permitted, I would like to have done several rounds of interviews
with enough time in between sessions to allow users to practice.

\paragraph{}

System performance has not been thoroughly tested as
performance was not a major concern in this iteration.
It is therefore a possibility that some of the solutions proposed
in this thesis may have scalability issues.
In order to ensure this is not the case, further development
of the prototype would be required.

\paragraph{}

The capabilities of LSP for the use case of vocal programming has been explored
during this thesis, but only partially tested.
Examples shown in Chapter~\ref{the_project} were created in isolation
from the prototype, and is therefore not been shown to work well
as part of the system.
Future iterations of the prototype would integrate LSP
to enable project wide awareness of symbols.

% Longevity
% Numbers of participants
% did not use lsp
% stress test
% hole oriented editing
